\documentclass{VRARWorkshop}

\usepackage[utf8]{inputenc}
\usepackage[T1]{fontenc}
\usepackage{ngerman}
\usepackage[hidelinks]{hyperref}

\title{Creating an augmented reality system to support manual non-destructive ultrasonic testing of metal pipes and plates}

\authors{Robert Deppe, Oliver Nemitz, Jens Herder}

\affiliations{~}
% TODO: Correct affiliations

\abstract{
The aim of this thesis is to describe the application of augmented reality technology in non–destructive testing of products of the metal–industry and to create a prototype.
This prototype is created with hard– and software, that is usually employed in the gaming industry, and delivers positions for creating c– scans.
Using the ZEDmini in combination with the HTC VIVE enables realtime visualisation of the probes path in the HMD, as well as the setting of virtual markers on the specimen.
As a part of the implementation the downhill–simplex optimization–algorithm is implemented to fit the specimen to a cloud of recorded surfacepoints.
The accuracy is statistically tested and evaluated with the result, that the VIVE–trackingsystem is accurate up to ca. 1–2 millimeters in well lit conditions.
This thesis is of interest not only for research–institutes of the metal–industry, but also for any areas of work, in which the enhancement with augmented–reality is possible.
}

% Give some keywords
\keywords{
Nondestructive Testing,
Ultrasonic,
Augmented Reality,
Tracking,
Stereocamera,
HTC VIVE,
ZED--mini,
NDT,
ZfP,
AR
}

\begin{document}

\section{Introduction}

\section{Related Research}
\cite{ARPat15}
\cite{ARClean}
\cite{schwerdtfeger_using_2008}
\cite{fadzil_design_2015}
\cite{walter_non-contact_2007}

\section{Basics of non-destructive ultrasonic testing of metal products}
\cite{deutsch_zfp_2010}
\cite{moles_introduction_2004}
\cite{olympus_Grundlagen}

\textcolor{red}{Soll noch ein Abschnitt zu den AR Grundlagen gemacht werden?}

\section{Description of the AR--application}

\section{Implementation}
\cite{dorner_virtual_2013}

\subsection{Tracking of the ultrasonic probe}

\subsection{Detection of the specimen geometry in worldspace}

\subsection{Vizualisation of the trackingdata}

\subsection{Continuous and singular logging of position}

\section{Evaluation}

\section{Conclusion}

\VRARsetbibstyle
\bibliography{VRARTeXample}

\end{document}
